% 사용 가능한 옵션 리스트는 아래와 같습니다.
% 'doctor - master': 박사인가 석사인가
% 'final': 최종본일 경우 추가
% 'twosides - oneside': 양면인가 단면인가
%
% 예제
% \documentclass[master,final,oneside]{KUThesis}

% 기본옵션은 [doctor,twosides] 입니다.
\documentclass{KUThesis}

%%%%%%%%%%%%%%%%%%%%%%%%%%%%%%%%%%%%%%%%%%%%%%%%%%%%%%%%%%%%%%%%
%                   아래 정보를 꼭 바꾸세요!                   %
%%%%%%%%%%%%%%%%%%%%%%%%%%%%%%%%%%%%%%%%%%%%%%%%%%%%%%%%%%%%%%%%

% 첫번째 인수의 숫자
% 0 : 제목이 10자 이하일 경우
% 1 : 제목이 10자 이상 1줄일 경우
% 2 : 제목이 2줄 이상일 경우
\title{2}{
Research on the fact that I cannot say ``father'' to the real father
}

% 본인의 이름 성,이름 순으로 한글자씩
\author[chinese]{洪}{吉}{同}
\author[english]{Hong}{Gil}{Dong}

% 지도교수님 성함 {한자이름}{영문이름}
\advisor{洪판서}{Pahn Seo Hong}

% 전공
\department{PH}

% 졸업년월
\graduateDate{2016}{2}

% 제출 연,월,일
\submitDate{2015}{12}{30}

%%%%%%%%%%%%%%%%%%%%%%%%%%%%%%%%%%%%%%%%%%%%%%%%%%%%%%%%%%%%%%%%
%                          여기까지!                           %
%%%%%%%%%%%%%%%%%%%%%%%%%%%%%%%%%%%%%%%%%%%%%%%%%%%%%%%%%%%%%%%%

\begin{document}

%%%%%%%%%%%%%%%%%%%%%%%%%%%%%%%%%%%%%%%%%%%%%%%%%%%%
%         본문 파일을 이 아래에 추가하세요!        %
%%%%%%%%%%%%%%%%%%%%%%%%%%%%%%%%%%%%%%%%%%%%%%%%%%%%

% 꼭 addContents함수를 이용해서 추가해야 양면출력이 제대로 됩니다!
\addContents{introduction}
\addContents{conclusion}

%%%%%%%%%%%%%%%%%%%%%%%%%%%%%%%%%%%%%%%%%%%%%%%%%%%%
%                  여기까지 추가                   %
%%%%%%%%%%%%%%%%%%%%%%%%%%%%%%%%%%%%%%%%%%%%%%%%%%%%

\end{document}
